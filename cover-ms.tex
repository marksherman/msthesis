% Cover pages for UMass Lowell Master's Thesis
% Revision 0.1  2010/11/27

%These need to be defined for the abstract page later on
\title{Exploration of Natural Design Strategies of Novice Engineers}
\author{Mark A. Sherman}
\prevdegrees{B.S., University of Massachusetts Lowell (2008)}
\department{Department of Computer Science}
% If the thesis is for two degrees simultaneously, list them both
% separated by \and like this:
% \degree{Doctor of Philosophy \and Master of Science}
\degree{Master of Science}
\degreemonth{December}
\degreeyear{2010}
\thesisdate{December 24, 2010}
% If there is more than one supervisor, use the \supervisor command
% once for each.
\supervisor{Fred G. Martin}{Associate Professor}
\committee{Michelle Scribner-Maclean}{Assistant Professor}


% Beginning of the manually constructed title page
\begin{titlepage}
% Month and year, plain text
\raggedright Winter 2010
\begin{center}

% Title of project
{\large \MakeUppercase{Exploration of Natural Design Strategies of Novice Engineers}}

BY

% Author and previous degrees
\MakeUppercase{Mark A. Sherman} \\
\MakeUppercase{B.S., University of Massachusetts Lowell (2008)}

\MakeUppercase{Submitted in partial fulfillment of the requirements} \\
\MakeUppercase{for the degree of master of science} \\
\MakeUppercase{Department of Computer Science} \\
\MakeUppercase{University of Massachusetts Lowell}

\end{center}

Signature of \\ 
\begin{tabular}{r p{7cm} c p{4cm} }
Author: &  & Date: &  \\ \cline{2-2} \cline{4-4}
\end{tabular}

Signature of Thesis \\ Supervisor: \underline{\hfill} \\
Name Typed: Fred G. Martin

Signatures of Other Thesis Committee Members:

Committee Member Signature: \\
Name Typed: Michelle Scribner-Maclean

\end{titlepage}

% Make the titlepage based on the above information.  If you need
% something special and can't use the standard form, you can specify
% the exact text of the titlepage yourself.  Put it in a titlepage
% environment and leave blank lines where you want vertical space.
% The spaces will be adjusted to fill the entire page.  The dotted
% lines for the signatures are made with the \signature command.
%\maketitle

% The abstractpage environment sets up everything on the page except
% the text itself.  The title and other header material are put at the
% top of the page, and the supervisors are listed at the bottom.  A
% new page is begun both before and after.  Of course, an abstract may
% be more than one page itself.  If you need more control over the
% format of the page, you can use the abstract environment, which puts
% the word "Abstract" at the beginning and single spaces its text.

%% You can either \input (*not* \include) your abstract file, or you can put
%% the text of the abstract directly between the \begin{abstractpage} and
%% \end{abstractpage} commands.

%%\newcounter{savepage}

% First copy: start a new page, and save the page number.
\newpage
\pagestyle{empty}
\mbox{}
\newpage
\pagestyle{empty}
\setcounter{savepage}{\thepage}

% Second copy: start a new page, and reset the page number.  This way,
% the second copy of the abstract is not counted as separate pages.

\newpage
\setcounter{page}{\thesavepage}
\begin{abstractpage}
% sherman 1
% $Log: abstract.tex,v $
% Revision 1.1  93/05/14  14:56:25  starflt
% Initial revision
% 
% Revision 1.1  90/05/04  10:41:01  lwvanels
% Initial revision
% 
%
%% The text of your abstract and nothing else (other than comments) goes here.
%% It will be single-spaced and the rest of the text that is supposed to go on
%% the abstract page will be generated by the abstractpage environment.  This
%% file should be \input (not \include 'd) from cover.tex.

This project explores how middle school students approach design problems, focusing on testing and iteration behaviors. Students were asked to solve design problems and create generalized processes for solving them. Observations of the students were analyzed using new methods for the characterization of testing and design iteration. These data yielded patterns of testing behavior intrinsic to the specific students, as well as patterns  within individual activities. From these patterns, guidelines for engineering activity design were created.

Students participated in five 60-minute activity sessions. Each session presented one activity that was focused on a specific engineering discipline. The disciplines include math problem solving, electrical engineering, mechanical engineering, and computer science. The subject areas that received treatment included parallel and series circuits, gear reduction and LEGO construction, algorithm design, real-time control systems, requirement satisfaction, and working within time constraints.

Students were video and audio recorded. Students were encouraged to talk through their process and were regularly prompted by investigators to verbalize their thoughts explicitly. The data were coded for important behaviors. From these codes, data on student iteration and testing patterns were extracted and analyzed. New methods of characterizing and analyzing testing and iteration were created. 

It was concluded that many factors were related to the success of the student design. The time spent exploring the problem before starting testing and iteration was the most significant factor. Other factors included the speed and consistency at which iterations were conducted. Recommendations for the creation of design-based engineering activities include scheduling techniques, introduction methods, and use of simulation tools.
\end{abstractpage}

%\newpage
%\section*{Acknowledgments}
%\centerline{\rule{5in}{.01in}}
%\chapter*{Acknowledgments}
\renewcommand{\thefootnote}{\fnsymbol{footnote}}

The work in this thesis was only possible because of all the helpful and guiding members of the UMass Lowell engineering education community. Many people have helped me not just in research effort, but in introducing me to new thoughts and ideas that became foundational to this project.

I would like to first thank Dr. Michelle Scribner-MacLean for being so inclusive towards me. She involved me in her work, where I learned about many of the foundational works that appear in this document. It was not just about getting sources for the literature review, it was about having an experienced colleague.  She taught me that everything is always a learning process, for students, teachers, and researchers alike.

I give my most sincere thanks to Dr. Fred Martin. He was patient and open-minded, yet always loyal to the science and process of development. I would like to thank Dr. Martin for years he has already invested in me, the care he has taken to see me succeed, and the all the opportunities he has afforded me. 

I thank Howard Sticklor, who set up the program upon which this research is based. I very much appreciate his support of both my work and that of our student participants.

I extend thanks to Dr. Sarah Kuhn and Michael Penta, with both of whom I have shared many exciting conversations on education methods and research.  Your ideas are woven into this thesis just as much as my own.

I thank my parents, Karen and Barry, who have been supportive and encouraging in all my endeavors, no matter how daunting. I believe that their continued reinforcement (and often reality checking)  has been critical in all of my successes. They sparked my interest in learning and teaching early on in my life. Their stories are inspirational to me, and I try every day to be as hard working as they are.

I would lastly like to thank my closest friends who have supported me through every phase of this process. Nick McKinnon, Mary Angeleri, and especially Stacy Kadesch, I thank you.

This material is based upon work supported by the National Science Foundation under Grants No. DRL-0624669 and No. DGE-0841392.\footnote{Any opinions, findings, and conclusions or recommendations expressed in this material are those of the author and do not necessarily reflect the views of the National Science Foundation.}

%%%%%%%%%%%%%%%%%%%%%%%%%%%%%%%%%%%%%%%%%%%%%%%%%%%%%%%%%%%%%%%%%%%%%%
% -*-latex-*-



