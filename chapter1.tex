\chapter{Introduction}

Engineering education lacks many of the well-developed and thoroughly-tested tools seen in more traditional educational subjects. In most areas of education there are volumes of methods for teaching and assessments measuring effectiveness of interventions. These mechanisms of instruction and assessment are generally well-tested in laboratory and field settings. Design and engineering lack such a body of work, as \citet[p. 35]{csed-guzdial} stated about the field of computing. Schools across the United States are adopting engineering curricula at nearly all levels of education, yet there are few research-supported mechanisms for understanding their effectiveness.

This study  intends to address this opportunity. The study was conducted in the Department of Computer Science in association with the Graduate School of Education. The activities presented were technical in nature. Design and analysis of these activities required deep domain knowledge in their respective engineering areas. The researchers possessed experience in these design fields, as well as knowledge in the field of education.

\section{Research Focus}
This study observed middle school students as they executed short engineering design activities. In the state of Massachusetts, middle school may include fifth through eighth grade. The students were introduced to engineering activities spanning multiple disciplines: electrical engineering, mechanical engineering, math problem solving, and computer science. The students were tasked to solve the given problems and create generalized processes for solving them. The focus of the data analysis was on testing, improvement, and iteration of the designs. New methods of characterizing and analyzing design iteration were created.
 
\section{Problem Statement}
This study is an exploration, and as such is interested multiple, closely related dimensions. The four questions of interest are:
\begin{itemize}
\item Do students exhibit patterns in testing and iteration? What are those patterns?
\item What characteristics of a design activity elicit specific iteration patterns?
\item What is the correlation between iteration in designing and success of the design?
\item What guidelines can be written for the creation of future activities?
\end{itemize}


\section{Approach}
This research used a laboratory study of a small sample population of middle school students. Middle school students are inherently novice designers, where they are highly unlikely to have received any previous formal engineering training. The students participated in five activities, one per week. Each activity represented a different discipline of engineering. Problem difficultly increased over the five weeks. The students were instructed and coached in thinking aloud to gain access to their tacit, internal processes. The sessions were video and audio recorded. The data was coded for many different design behaviors, informed by \citep{welch}. From this data information about testing and iteration was extracted, and methods of characterizing testing and iteration were developed.

\section{Hypothesis and Contributions}
Here, each research question is framed in more detail:
\subsubsection{Do students exhibit patterns in testing and iteration?}
	It is expected that individual students will demonstrate personal trends across all activities. 
	
\subsubsection{What characteristics of a design activity elicit specific iteration patterns?}
	The design activities were designed to differ in complexity, speed of construction, and level of abstractness. These types of properties are expected to have a specific effect on iteration patterns in students, resulting in each activity having general trends that cross all students within the specific activity.
	
\subsubsection{What is the correlation between iteration in design the design process and the success of the final design?}
	Multiple sources in the literature depict iteration as critical to design success. For example, \citet{dow09} showed that, in college students, forced iteration makes an inexperienced designer just as good as non-iterating designer who has domain experience. In this study, it was expected that rate and count of iteration would strongly correlate with success. Each activity is analyzed with individual success metrics, so this hypothesis was tested within each activity separately. 

\subsubsection{What guidelines can be written for the creation of future activities?}
	Each activity was expected to result in a certain unique pattern of testing and iteration. By comparing these patterns, it would then be possible to generate recommendations on properties of the activities themselves. These guidelines could be generalized for use by educators.

\section{Rationale}
Exploring the tacit processes of novice engineers will further our understanding of human design faculties. With this study, motivations behind why students engage in certain models of behavior was explored. This study chose to focus on iteration, a single component of the engineering process. As is discussed in the Background chapter, iteration is a fundamental characteristic of the design process. This study chose to use design iteration as a lens for analysis of the student design work.



%\section{Organization}TODO 
