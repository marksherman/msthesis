% sherman 1
% $Log: abstract.tex,v $
% Revision 1.1  93/05/14  14:56:25  starflt
% Initial revision
% 
% Revision 1.1  90/05/04  10:41:01  lwvanels
% Initial revision
% 
%
%% The text of your abstract and nothing else (other than comments) goes here.
%% It will be single-spaced and the rest of the text that is supposed to go on
%% the abstract page will be generated by the abstractpage environment.  This
%% file should be \input (not \include 'd) from cover.tex.

This project explores how middle school students approach design problems, focusing on testing and iteration behaviors. Students were asked to solve design problems and create generalized processes for solving them. Observations of the students were analyzed using new methods for the characterization of testing and design iteration. These data yielded patterns of testing behavior intrinsic to the specific students, as well as patterns  within individual activities. From these patterns, guidelines for engineering activity design were created.

Students participated in five 60-minute activity sessions. Each session presented one activity that was focused on a specific engineering discipline. The disciplines include math problem solving, electrical engineering, mechanical engineering, and computer science. The subject areas that received treatment included parallel and series circuits, gear reduction and LEGO construction, algorithm design, real-time control systems, requirement satisfaction, and working within time constraints.

Students were video and audio recorded. Students were encouraged to talk through their process and were regularly prompted by investigators to verbalize their thoughts explicitly. The data were coded for important behaviors. From these codes, data on student iteration and testing patterns were extracted and analyzed. New methods of characterizing and analyzing testing and iteration were created. 

It was concluded that many factors were related to the success of the student design. The time spent exploring the problem before starting testing and iteration was the most significant factor. Other factors included the speed and consistency at which iterations were conducted. Recommendations for the creation of design-based engineering activities include scheduling techniques, introduction methods, and use of simulation tools.